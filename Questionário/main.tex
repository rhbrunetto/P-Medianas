% Pacotes
\documentclass[12pt]{article}
\usepackage{adjustbox}
\usepackage[utf8]{inputenc}
\usepackage{amsmath}
\usepackage{hyperref}
\usepackage{sbc-template}
\usepackage{fancyvrb}
\usepackage{amsfonts}
\usepackage{amsmath}
\usepackage{graphicx,url}


\sloppy

\title{3ª Avaliação de 2017 – Parte 2\\Modelagem e Otimização Algorítmica}

\author{Ricardo Henrique Brunetto (RA94182)\inst{1}}


\address{Departamento de Informática -- Universidade Estadual de Maringá (UEM)\\
	Maringá -- PR -- Brasil
	\email{ra94182@uem.br}
}

\begin{document}

	\maketitle

	%{\resumo{}}

  \section*{Questão 01}
	Os algoritmos heurísticos podem ser divididos em \textit{Algoritmos Construtivos} e \textit{Algoritmos Melhorativos},
	sendo:
	\begin{itemize}
		\item \textbf{Algoritmo Construtivo}: Sua função é construir inteiramente a solução
		partindo da heurística em questão. Nesse caso, não considera-se nenhuma parte da solução
		já construída, ou seja, o método assume como estado inicial a \textbf{solução vazia}.
		Em seguida, aplica-se a heurística iterativamente, adicionando um novo elemento a
		cada iteração até que atinja determinado critério de parada e obtenha a solução.

		\item \textbf{Algoritmo Melhorativo}: Sua função é, dada uma solução inicial,
		obter uma solução de melhor valor de função objetivo. Pode-se, inclusive, fazer uso
		de outra heurística para cálculo do valor de solução. Um algotimo melhorativo
		faz uso de uma \textbf{Busca Local} para percorrer soluções na vizinhança
		e, através de determinado critério de parada, obter uma solução melhor.
	\end{itemize}

	\section*{Questão 02}
	A Solução Ótima (Global) de um problema está vinculada ao melhor valor da função-objetivo
	para todas as possíveis combinações das variáveis no âmbito que se considera. Por outro lado,
	a Solução Local (Ótimo Local) é referente ao melhor valor da função-objetivo em determinada
	vizinhança (localidade) que se considera. Dessa forma, enquanto um valor é referente à melhor
	solução que se pode obter em todos os cenários, outro valor é referente à melhor solução que
	se pode obter em determinada região de análise.\\
	Obter um ótimo local ou ótimo global é dependente da heurística que se adota. Métodos exatos
	e algumas heurísticas são capazes de obter o ótimo global para determinados problemas. Contudo,
	principalmente quando se envolve otimização combinatória, obter a solução ótima torna-se computacionalmente
	trabalhoso na maior parte dos casos. Assim, o ótimo local se sai como a melhor solução encontrada na
	vizinhança, enquanto o ótimo global é a melhor dentre todos os ótimos locais.\\
	Dessa forma, a qualidade de uma solução relaciona-se diretamente com a vizinhança onde se
	deseja buscar e a heurística que se adota. Assim, muitas vezes é possível obter uma solução local
	para determinado problema que seja satisfatória para os parâmetros dados e tempo computacional
	esperado.

	\section*{Questão 03}
	Uma solução $x$ é dita \textbf{vizinha} de uma solução $y$ quando é possível transformar $x$ em
	$y$ através da aplicação de uma determinada operação, chamada de \textbf{movimento}, que executa
	uma perturbação elementar em $x$. Em outras palavras, uma solução é vizinha de outra quando é possível alterar seus elementos seguindo
	determinada operação. A busca local consiste em "caminhar" entre as soluções vizinhas de determinada solução para
	procurar um ótimo local e, então, gerar as soluções vizinhas deste. É extremamente útil fazer uso
	da técnica de encontrar soluções vizinhas para se restringir o espaço de busca através de determinada
	heurística.\\
	Dessa forma, a vizinhança de uma solução é constituída por todas as soluções que se pode encontrar
	aplicando \textbf{um movimento}. Isso faz com que se obtenha soluções derivadas uma única vez. Em
	algoritmos heurísticos, conforme citado, faz-se uso da vizinhança para aplicar a heurística a determinado
	espaço de busca. Assim, adota-se uma estratégia de busca na vizinhança (Busca Local) e aplica-se
	a heurística para encontrar o valor da função-objetivo. Logo, aliado à estratégia de busca, tem-se o
	ótimo local para o problema através da aplicação da heurística.

	\section*{Questão 04}
	Existem diversas formas de explorar a vizinhança de uma solução. Tais formas são chamadas de estratégias de busca, e servem para "caminhar" entre as soluções através de um critério de modificação da solução para gerar a vizinhança. Existem duas formas clássicas, dentre as variações:
	\begin{itemize}
		\item \textbf{First Improvement}, ou Melhoria Iterativa, onde é selecionada uma solução qualquer
		da vizinhança que apresente melhoria. Em geral, seleciona-se a primeira solução gerada que resulte
		em melhora no valor da função-objetivo. Há economia de processamento e tempo computacional, visto que não se torna necessário gerar todas as soluções da vizinhança ou realizar uma análise mais detalhada das mesmas. Contudo, pode ser que a primeira solução com melhoria gerada não seja tão expressiva quanto outra futura que não fora considerada.
		\item \textbf{Best Improvement}, ou Descida mais Rápida, tende a selecionar a melhor solução da vizinhança, ou seja, aquela que apresenta uma melhoria mais expressiva. Para tanto, é necessário calcular todas as soluções envolvidas na vizinhança e analisar as melhorias proporcionadas por cada uma, o que toma mais tempo computacional e processamento. Contudo, o nome "Descida mais Rápida" refere-se ao fato de que tal estratégia seleciona sempre as melhores opções das vizinhanças que percorre (ótimos locais), o que proporciona uma conversão mais acelerada a determinado valor de solução final.
	\end{itemize}

	\section*{Questão 05}
	Em termos gerais, \textbf{Meta-Heurística} pode ser definida como uma estratégia geral para projetar procedimentos heurísticos com alta performance. De forma geral, uma meta-heurística se propõe a basear um modelo para construção de uma heurística, que visa a transpor falhas da busca local, por exemplo. Assim, uma meta-heurística é, a grosso modo, uma heurística para criar heurísticas. Em geral, baseia-se em modelos da natureza, como fenômenos físicos, quimicos e biológicos.\\
	Dentre as diversas meta-heurísticas existentes, as mais clássicas são:
	\begin{itemize}
		\item \textbf{Algoritmos Genéticos}: consiste de um algoritmo probabilistico com base análoga no processo de evolução natural, através da seleção de soluções de um problema de otimização combinatória. Em suma, mantém-se uma população onde cada indivíduo é um cromossomo (solução) e, então, selecionam-se cromossomos reprodutores para gerar uma nova população através de um cruzamento e posterior possível mutação, inserindo novos elementos e removendo outros.
		\item \textbf{Simulated Annealing}: consiste de um processo onde se faz analogia com o processo industrial de \textit{annealing}. Assim, parte-se de uma alta temperatura (solução longe do ótimo local) e resfriar o sistema (refinar a solução). Para isso, admite-se uma probabilidade de construir soluções ruins em altas temperaturas e, conforme o sistema vai se resfriando, tal probabailidade diminui, visto que se aproxima do ótimo local.
		\item \textbf{Ant System}: consiste de um paradigma computacional que propõe soluções com base em uma analogia com o comportamento de colônias de formigas naturais. Nesse ínterim, simulam-se formigas (chamadas de formigas artificiais) para servirem de ferramenta de otimização. Assim, tem-se uma população de formigas artificiais, onde cada qual formulará uma solução (o caminho percorrido). Dentre diversos fatores, um dos agentes mais importantes nesse contexto é o feromônio, tratado como uma forma de auto-reforçar o comportamento das formigas e conduzir os valores de probabilidade para percorrer determinado caminho. Como consequência, estima-se uma convergência mais rápida para solução (não é regra).
	\end{itemize}

	\section*{Questão 06}
	Não. Em geral, a aplicação clássica de um Algoritmo Genético comporta apenas as \textbf{operações} que o caracterizam. As operações são o que definem o Algoritmo Genético básico, sendo elas, em geral:
	\begin{itemize}
		\item \textbf{Seleção}: selecionar os reprodutores (indivíduos a serem cruzados);
		\item \textbf{Reprodução}: consiste em executar uma combinação dos pais para gerar filhos;
		\item \textbf{Mutação}: uma taxa (pequena) da população é submetida a uma ligeira modificação arbitrária.
	\end{itemize}
	Em geral, tem-se uma avaliação de aptidão dos cromossomos filhos para ter base da qualidade da solução gerada e, em seguida, faz-se a seleção dos reprodutores, cruzando-os e aplicando a mutação aos novos indivíduos, quando válido. Por fim, atualiza-se a população.\\
	Contudo, a busca local, que se encaixa após a geração de um novo indivíduo, pode colaborar para a convergência do algoritmo uma vez que se considera sempre o ótimo local em relação à vizinhança de um indivíduo gerado, em vez dele próprio.

	\section*{Questão 07}
	Em algoritmos heurísticos, o termo \textbf{estagnação} representa uma melhoria inexpressiva (ou inexistente) entre sucessivas soluções. Em geral, a estagnação pode significar que se encontrou um ótimo (local ou global) ou que houve uma convergência precoce - e errônea - do algoritmo, o que é mais comum.\\
	Em termos analíticos, uma meta-heurística com maiores chances de estagnar é o \textbf{Ant System} (Sistema Fórmico). Isso porque o algoritmo faz uso de uma estratégia de reforço (feromônio) para determinadas soluções baseando-se em probabilidade, o que pode, após determinadas iterações, induzir fortemente determinados caminhos, em detrimento excessivo a outros. Dessa forma, não há diversidade de soluções e acaba-se por não explorar outras possibilidades de solução, o que faz o algoritmo se voltar apenas para determinada vertente, estagnando.\\
	Em algoritmos genéticos, a estagnação pode ser evitada através da manutenção de uma diversidade maior dentro da população, ou seja, quanto mais diferentes os indivíduos forem entre si, mais diversificados serão os indivíduos gerados e, dessa forma, tende-se a manter uma população com diferentes níveis. Nesse íterim, há ênfase para o \textit{crossover} (reprodução) entre os cromossomos e para a mutação. Além disso, apenas gerar indivíduos novos - e diversificados - não é o suficiente, devendo-se adotar uma estratégia coerente para atualização da população. Nesse caso, é interessante manter parte da população antiga ao adotar a nova, uma vez que preseva-se os genes mais perpetuantes bem como os mutados.

	\section*{Questão 08}
	Em relação às meta-heurísticas que trabalham com um conjunto de soluções, tem-se como exemplo:
	\begin{itemize}
		\item \textbf{Algoritmos Genéticos}: mantém uma população de soluções que são combinadas e geram novas soluções, seguindo os operadores característicos de um algoritmo genético.
		\item \textbf{Ant System}: mantém uma população de soluções, uma em cada formiga, gerando novas soluções conforme as iterações do algoritmo e a distribuição de feromônio.
	\end{itemize}
	Por outro lado, em relação às meta-heurísticas que operam em uma única solução, tem-se:
	\begin{itemize}
		\item \textbf{Simulated Annealing}: manipula uma solução de maneira análoga ao processo de resfriamento de cristais, iniciando com altas temperaturas (solução longe do ideal) e diminuições bruscas até atingir uma temperatua mais baixa (solução próxima do ideal) e operar diminuições paulatinas.
		\item \textbf{GRASP}: (\textit{Greedy Randomized Adaptative Search Procedure}) opera em uma única solução, melhorando-a através de uma busca local.
	\end{itemize}

	\section*{Questão 09}
	É recomendada a meta-heurística \textbf{Variable Neighborhood Search}. Isso porque ela possibilita de explorar sistematicamente múltiplas vizinhanças de uma mesma solução de modo dinâmico. Nesse contexto, tal meta-heurística explora a melhor solução de cada uma das vizinhanças que se trabalha.\\
	O processo se dá através da tentativa de aprimorar a solução através das sucessivas vizinhanças que se tem disponível. Dessa forma, tem-se mais de uma forma de explorar as soluções derivadas e maior chance de encontrar a solução ótima.

	\section*{Questão 10}
	Em relação a um processo adaptativo, tem-se a \textbf{Busca-Tabu}.\\
	Seu processo de aprendizado se dá através da inibição de movimentos que não levam a nova soluções. Assim, quando se calcula determinada solução, é utilizada uma \textbf{estrutura de memória} para armazená-la. O mais comum é utilizar uma \textbf{Lista Tabu}, que compõe uma lista das soluções consideradas proibidas por determinado número de iterações (\textbf{Prazo Tabu}), visto que futuramente podem ser úteis. De posse dessa estrutura, calcula-se, a cada iteração, uma solução vizinha que seja melhor que a corrente, descartando soluções que já estão contidas na lista.\\
	Dessa forma, evita-se cometer "erros" ao utilizar um mecanismo de memória e adaptatividade para armazenar soluções já visitadas.

	\bibliographystyle{sbc}
	\bibliography{references}
	\nocite{*}

\end{document}
