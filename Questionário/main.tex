% Pacotes
\input{Preambulo-sbc.tex}

\sloppy

\title{Questionário - Prova 03\\Modelagem e Otimização Algorítmica}

\author{Ricardo Henrique Brunetto (RA94182)\inst{1}}


\address{Departamento de Informática -- Universidade Estadual de Maringá (UEM)\\
	Maringá -- PR -- Brasil
	\email{ra94182@uem.br}
}

\begin{document}

	\maketitle

	{\resumo{}}

  \section*{Questão 01}
	Os algoritmos heurísticos podem ser divididos em \textit{Algoritmos Construtivos} e \textit{Algoritmos Melhorativos},
	sendo:
	\begin{itemize}
		\item \textbf{Algoritmo Construtivo}: Sua função é construir inteiramente a solução
		partindo da heurística em questão. Nesse caso, não considera-se nenhuma parte da solução
		já construída, ou seja, o método assume como estado inicial a \textbf{solução vazia}.
		Em seguida, aplica-se a heurística iterativamente, adicionando um novo elemento a
		cada iteração até que atinja determinado critério de parada e obtenha a solução.

		\item \textbf{Algoritmo Melhorativo}: Sua função é, dada uma solução inicial,
		obter uma solução de melhor valor de função objetivo. Um algotimo melhorativo
	\end{itemize}

	\bibliographystyle{sbc}
	\bibliography{references}

\end{document}
